\documentclass[sigconf]{acmart}

\usepackage{xeCJK}
\usepackage{subfigure}
\usepackage{graphicx}
\usepackage{array}
\usepackage{enumitem}
\usepackage{multicol}
\usepackage{algorithm,algorithmic}
\usepackage{color}  
%\definecolor{shadecolor}{named}{Gray}  
\definecolor{shadecolor}{rgb}{0.92,0.92,0.92}  
\usepackage{framed}
\usepackage{ulem}

%% Font
\CJKfontspec{Noto Serif CJK TC} %思源宋體

%% ----------
\begin{document}

\title{離散數學HW05}

\author{易頡~110054809}
\orcid{}
\affiliation{%
  \institution{隨班附讀}
  \city{}
  \country{}
}

%% 刪除ACM Reference Format信息
\settopmatter{printacmref=false} % Removes citation information below abstract
\renewcommand\footnotetextcopyrightpermission[1]{} % removes footnote with conference information in first column
\pagestyle{plain} % removes running headers

\maketitle

%% ----- Question -----
\section{Question}
\begin{itemize}
	\item[-] page 350, chapter 5.1 Exercise 6
	\item[-] page 363, chapter 5.2 Exercise 6
	\item[-] page 379, chapter 5.3 Exercise 26 
	\item[-] page 391, chapter 5.4 Exercise 8
\end{itemize}

%% ----- Problem -----
\section{Answer}
\subsection{page 350, chapter 5.1 Exercise 6}
\begin{shaded}
    Prove that $1 \cdot 1! + 2 \cdot 2! + ... + n \cdot n! = (n + 1)! - 1$ whenever n is a positive integer.
\end{shaded}  
Prove ~~~$1 \cdot 1! + 2 \cdot 2! + ... + n \cdot n! = (n + 1)! - 1$ \\
For all $n \geq 1 $, $ P(n) = (n + 1)! - 1 $ \\
Basis step:\\
$n = 1$, $ P(1) $ is true, since ~~~ $ 1 \cdot 1! = (1 + 1)! - 1 = (2)! - 1 $ \\
Inductive step:\\
Inductive hypothesis, assume $ P(k) $ holds for an arbitrary positive integer $k$. ~~~  $1 \cdot 1! + 2 \cdot 2! + ... + k \cdot k! = (k + 1)! - 1 $\\
Add $ (k + 1) \cdot (k + 1)!$ to both sides of the equation in $ P(k) $, we obtain
\begin{eqnarray*}
1 \cdot 1!  + ... + k \cdot k! + (k + 1) · (k + 1)!  & = & (k + 1)! - 1 + (k + 1) · (k + 1)!\\
~ & = & (k + 1)!(\uline{~~1+k+1~~}) - 1 \\
~ & = & \uline{~~(k+2)!-1~~} 
\end{eqnarray*}
By mathematical induction, $ P(n) $ is true for all integer $n$ with $n \geq 1 $.

\subsection{page 363, chapter 5.2 Exercise 6(a)}
\begin{shaded}
    Determine which amounts of postage can be formed using just 3-cent and 10-cent stamps.
\end{shaded}  
We can form the following amounts of postage as indicated:\\
$ 3 = 3$\\
$ 6 = 3 + 3$\\
$ 9 = 3 + 3 + 3$\\
$10 = 10$\\
$12 = 3 + 3 + 3 + 3$\\
$13 = 10 + 3$\\
$15 = 3 + 3 + 3 + 3 + 3$\\
$16 = 10 + 3 + 3$\\
$18 = 3 + 3 + 3 + 3  + 3$\\
$19 = 10 + 3 + 3 + 3$\\
$20 = 10 + 10$\\
By having considered all the combinations, we know that the gaps in this list cannot be filled.\\ We claim that we can form all amounts of postage greater than or equal to \uline{~~18~~} cents using just 3-cent and 10-cent stamps.
    
\subsection{page 363, chapter 5.2 Exercise 6(b)}
\begin{shaded}
    Prove your answer to (a) using the principle of mathematical induction. Be sure to state explicitly your inductive hypothesis in the inductive step.
\end{shaded}
Let $P(n)$ be the statement that we can form n cents of postage using just 3-cent and 10-cent stamps. We want to prove that $P(n)$ is true for all $n \geq 18$.\\
Basis step:\\
$n = 18 = 3 + 3 + 3 + 3 + 3 + 3$. \\
Inductive step:\\
Assume that we can form $k$ cents of postage (the inductive hypothesis); we will show how to form $k + 1$ cents of postage. If the $k$ cents included two 10-cent stamps, then replace them by \uline{~~seven~~} 3-cent stamps (7 · 3 = 2 · 10 + 1).\\
Otherwise, $k$ cents was formed either from just 3-cent stamps, or from one 10-cent stamp and $k-10$ cents in 3-cent stamps. Because $k \geq 18$, there must be at least \uline{~~three~~} 3-cent stamps involved in either case. Replace \uline{~~three~~} 3-cent stamps by \uline{~~one~~} 10-cent stamp, and we have formed $k + 1$ cents in postage (10 = 3 · 3 + 1).

\subsection{page 363, chapter 5.2 Exercise 6(c)}
\begin{shaded}
    Prove your answer to (a) using strong induction. How does the inductive hypothesis in this proof differ from that in the inductive hypothesis for a proof using mathematical induction?
\end{shaded}
Let $P(n)$ be the statement that we can form n cents of postage using just 3-cent and 10-cent stamps. We want to prove that $P(n)$ is true for all $n \geq 18$. \\
Basis step:\\ To prove that $P(n)$ is true for all $n \geq 18$, we note for the basis step that from part (a), $P(n)$ is true for $n = 18, 19, 20$. \\
Inductive step:\\ Assume the inductive hypothesis, that $P(j)$ is true for all $j$ with $18 \leq j \leq k$ , where $k$ is a fixed integer greater than or equal to 20. We want to show that $P(k + 1)$ is true. Because $k - 2 \geq 18$, we know that $P(k - 2)$ is true, that is, that we can form $k - 2$ cents of postage. Put one more \uline{~~3~~}-cent stamp on the envelope, and we have formed $k + 1$ cents of postage, as desired.\\
In this proof our inductive hypothesis included all values between 18 and $k$ inclusive, and that enabled us to jump back three steps to a value for which we knew how to form the desired postage.

\subsection{page 379, chapter 5.3 Exercise 26}
\begin{shaded}
    Let $\textsl{S}$ be the set of positive integers defined by\\
    $\textsl{Basis step}$ : $1 \in \textsl{S}$.\\
    $\textsl{Recursive step}$ : If $n \in \textsl{S}$, then $3n + 2 \in \textsl{S}$ and $n^2 \in \textsl{S}$.
     \begin{enumerate}[label=(\alph*)]
        \item Show that if $n \in \textsl{S}$, then $n \equiv 1$ (mod 4)
        \item Show that there exists an integer $m \equiv 1$ (mod 4) that does not belong to \textsl{S}
    \end{enumerate}
\end{shaded}
\begin{enumerate}[label=(\alph*)]
    \item \uline{~~The basis step is the observation that $1 \equiv 1(mod4)$. For the inductive step, if $n \equiv 1(mod4)$, then $3n+2 \equiv 3 \cdot 1+2 \equiv 1(mod4)$ and $n^2 \equiv 12 = 1(mod4)$.~~}
    \item \uline{~~One example is that $9 \notin S$ . Because 9 is not of the form 3n+2, the only way 9 could have gotten into S would be via 9=32, but $3 \notin S$ because $3 \not\equiv 1(mod4)$.~~}
\end{enumerate}

\vspace{15cm}

\end{document}
\endinput
%%
%% End of file `sample-sigconf.tex'.
